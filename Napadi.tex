\documentclass[a4paper]{article}

\usepackage{color}
\usepackage{url}
\usepackage[utf8]{inputenc} % make weird characters work
\usepackage{graphicx}

\usepackage[english,serbian]{babel}
%\usepackage[english,serbianc]{babel} %ukljuciti babel sa ovim opcijama, umesto gornjim, ukoliko se koristi cirilica
\setcounter{tocdepth}{1}
\usepackage[unicode]{hyperref}
\hypersetup{colorlinks,citecolor=green,filecolor=green,linkcolor=blue,urlcolor=blue}
\usepackage{amsmath}
\usepackage{amssymb}

\begin{document}

\title{Napadi na supersingularne kriptografske sisteme umetanjem greške\\ \small{~\\Seminarski rad u okviru kursa Kriptografija\\ Prof. Miodrag Živković\\Matematički fakultet}}

\author{Bukurov Anja}
\date{27.~maj 2017.}
\maketitle

\abstract{
Predstavljamo prvi napad umetanjem greške na kripto sisteme koji su zasnovani na supersingularnim izogenijima. Tokom računanja pomoćnih tačaka, napad cilja da promeni početnu tačku nekom slučajno odabranom tačkom na krivoj umetanjem greške. Pokazaćemo da će ovakav napad otkriti tajnu izogeniju jednom uspešnom smetnjom koja ima visoku verovatnoću uspeha.}


\section{Uvod}

Kriptografske sisteme zasnovane na izogenijama između supersingularnih eliptičkih krivih predložili su Jao i De Feo 2011. godine kao kandidat za kriptografski protokol. Umesto da se oslanja na problem diskretnog logaritma, koji je osetljiv na Šorov (engl. Shor) algoritam, ovaj protokol zasnovan je na problemu pronalaženja izogenija između supersingularnih eliptičkih krivih.

Napad umetanjem greške iskorišćava curenje osetljivih informacija kada implementacija radi pod neočekivanim uslovima. U ovom radu ispitujemo efekte promene tačke $P$ u neko slučajno odabranu tačku $P_0$ i pokušaj da se otkrije tajna, a to je u ovom slučaju izogenija $\phi$. Napad bi bio u mogućnosti da otkrije celu tajnu $\phi$ iz jednog izlaza $\phi(P_0)$ sa visokom verovatnoćom. Predstavićemo napad sa umetanjem greške u kontekstu nekoliko signaturnih shema i protokolu razmene ključa. Napad bi radio i protiv protivmera koje je predložio Kirkvud (engl. Krikwood) koje su zasnovane na Fudžisaki-Okamoto (engl. Fujisaki-Okamoto) transformaciji. Glavno zapažanje je da korisnik nikada ne treba da otkriva sliku slučajno izabranih tačaka pod tajnom izogenijom. 


\section{Supersingularni izogeni kriptografski sistem}

U ovom delu, upoznajemo se sa protokolom razmene ključa, interaktivnim protokolom identifikacije i raznim signaturnim shemama. 

\paragraph{Razmena ključa} Pretpostavimo da Alisa i Bob žele da uspostave zajedničku tajnu. Moraju izvršiti tri koraka kako bi to uradili:

\begin{description}
	\item[Priprema:] Bira se prost broj $p$ oblika $p = l^{e_A}_{A} \cdot l^{e_B}_{B} \cdot f \pm 1$ gde su $l_{A}$ i  $l_{B}$ različiti mali prosti brojevi, $f$ je mali kofaktor, dok su $e_A$ i $e_B$ pozitivni celi brojevi takvi da važi $l^{e_A}_{A} \approx l^{e_B}_{B}$. Zatim je potrebno odrediti supersingularnu eliptičku krivu $E$ u polju $\mathbb{F}_{p^2}$ i početne tačke \{$P_A, Q_A$\} i \{$P_B, Q_B$\}. 
	
	
	\item[Generisanje ključa:] Alisa slučajno odabira elemente $a_1, a_2 \in \mathbb{Z}/l^{e_A}_{A}\mathbb{Z}$, takve da nisu oba deljiva sa $l_A$ i određuje podgrupu $G_A = <h[a1]P_A + [a2]Q_A>$. Zatim koristi Veluovu (V\' elu) formulu da odredi krivu $E_A = E/G_A$ i izogeniju $\phi_A: E \longrightarrow E_A$, gde je $ker \phi_A = G_A$. Alisa takođe računa tačke $\phi_A(P_B)$ i $\phi_A(Q_B)$. Onda Bobu šalje torku $(E_A, \phi_A(P_B), \phi_A(Q_B))$. Bob  zatim vrši izračunavanja sa druge strane.
	
	\item[Poreklo ključa:] Kada primi Bobovu torku $(E_B, \phi_B(P_A), \phi_B(Q_A))$, Alisa određuje podgrupu $G_A' = <h[a1]\phi_B(P_A) + [a2]\phi_B(Q_A)>$ i koristi Veluovu formulu da odredi krivu $E_{AB} = E_B / G_A'$. Zatim koristi j-invarijantu krive $E_{AB}$ kao zajedničku tajnu. Bob postupa na isti način. Protokol rezimiramo na slici \ref{fig: razmena kljuca}.
	
\end{description}


\begin{figure}[h]
	\centering
	\includegraphics[width=0.8\textwidth]{razmena_kljuca.png}
	\caption{Protokol razmene ključa}
	\label{fig: razmena kljuca}
\end{figure}


\end{document}